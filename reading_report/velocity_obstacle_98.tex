\documentclass[12pt]{article}
\usepackage{natbib,amsmath,amsfonts,fullpage,hyphenat,booktabs,graphicx,setspace}
\usepackage[colorlinks,linkcolor=blue,citecolor=blue,urlcolor=blue]{hyperref}
%\onehalfspacing{}
\setcitestyle{square,comma,super}
\title{Reading report\\Motion planning in dynamic environment using velocity obstacle\cite{fiorini}}
\author{Dharmin Bakaraniya}
\begin{document}
\maketitle{}
\begin{abstract}
This paper presents a method for robot motion planning in dynamic
environments. It consists of selecting avoidance maneuvers to avoid
static and moving obstacles in the velocity space, based on the current positions and velocities of the robot and obstacles. It is a first-
order method, since it does not integrate velocities to yield positions
as functions of time.

The avoidance maneuvers are generated by selecting robot ve-
locities outside of the velocity obstacles, which represent the set of
robot velocities that would result in a collision with a given obstacle
that moves at a given velocity, at some future time. To ensure that
the avoidance maneuver is dynamically feasible, the set of avoidance
velocities is intersected with the set of admissible velocities, defined
by the robot’s acceleration constraints. Computing new avoidance
maneuvers at regular time intervals accounts for general obstacle
trajectories.

The trajectory from start to goal is computed by searching a tree of
feasible avoidance maneuvers, computed at discrete time intervals.
An exhaustive search of the tree yields near-optimal trajectories that
either minimize distance or motion time. A heuristic search of the
tree is applicable to on-line planning. The method is demonstrated
for point and disk robots among static and moving obstacles, and
    for an automated vehicle in an intelligent vehicle highway system
scenario.
\end{abstract}

\section{Summary}
\begin{itemize}
    \item Motion planning in dynamic environment is considerably more difficult that static motion planning because it needs to solve path planning and velocity planning at the same time.
    \item Motion planning in static environment is guaranteed to come up with a solution if one exists but in dynamic environment the solution is `intractable'.
    \item ``The velocity obstacle is a first-order approximation of the robot’s velocities that would cause a collision with an obstacle at some future time, within a given time horizon.''\cite{fiorini}
    \item Velocity obstacle (VO) is an extention of configuration space obstacle for a dynamic environment.
    \item For any given obstacle $B_1$ and moving object $A$, there will be a collision if $\bold{v}_{A,B_1}$ lies between the two tangents of $B_1$ ($\lambda_f$ and $\lambda_r$).
    \item The circle around which the tangents are to be drawn depend on the velocity of A. This makes a lot more sense because if A was moving very slowly then the circle around $B_1$ would be very small which means a small set of very specific movement of A will actually result into collision.
    \item The area between these tangents is called \itemit{collision cone}.
    \item Velocity obstacle is addition (Minkowski vector sum) of collision cone with the velocity $\bold{v}_{B_1}$ $$VO = CC_{A,B_1} + \bold{v}_{B_1}$$
    \item For stationary obstacle, $\bold{v}_{B_1} = 0$ and thus VO will not be translated.
    \item VO for multiple obstacle can be combined $$VO = \bigcup^{m}_{i=1}VO_i$$
    \item VO can be calculated periodically to fight against variable velocity.
    \item To avoid obstacle, the robot calculates it reachable velocities by considering the acceleration that it can achieve and the time till next interval.
    \item The authors state that the dynamic constraints are more restrictive than nonholonomic kinematic constraints when the robt is in motion.
    \item There can be at most 3 regions that the robot can be depending on its own velocity and the velocity of the obstacle namely \textit{front}, \textit{rear} and \textit{diverging}. All these regions are basically maneuvers that the robot can take to avoid colliding with obstacle.
    \item For multiple obstacles, the authors use the notation of $S_{string}$, where \textit{string} is ordered string of maneuvers (\textit{f,r,d}) that the robot can take. For example, $S_{fr}$ means the maneuver that will enable the robot to avoid first obstacle by going in \textit{front} of it and avoid the second obstacle by going in \textit{rear} of it.
    \item From this reachable velocities, the robot can then calculate reachable avoidance velocities (RAV) by substracting VO from reachable velocities.
    \item To calculate avoidance trajectories, authors suggest two approaches. 
        \begin{enumerate}
            \item Global search over all feasible meaneuvers at regular interval.
                \begin{itemize}
                    \item Global search is performed by creating a search tree.
                    \item To reduce the complexity, the authors suggest to use grids for each RAV\@.
                    \item Each depth level in the tree represents time.
                \end{itemize}
            \item Heuristic search for online application when the trajectories of obstacles are not known before hand. There are 2 hueristic proposed by the authors
                \begin{enumerate}
                    \item TG (to goal): Selects highest avoidance velocity in direction of goal
                    \item MV (maximum velocity): Selects maximum velocity towards the goal within $\alpha$ variance. 
                    \item ST (structure): Selects velocity according to obstacle's perceived risk.
                \end{enumerate}
        \end{enumerate}
    \item The heuristic approaches is a function which prioritises
        \begin{itemize}
            \item \textbf{Primary goal}: survival of robot
            \item \textbf{Secondary goal}: reaching the target, minimizing travel time and selecting desired trajectory structure
        \end{itemize} 
    \item The heuristic approaches could be combined or switched depending on the situation.
    \item An interesting fact is that if the time interval between 2 calculations is set very high, then there might not be any solutions as the whole RAV would be covered by VO\@.
    \item Experiments on real robot was perfromed on the highway exit ramp problem. Here obstacles are traveling at constant speed in a direction in their lanes. The robot has to reach an exit from the highway while avoiding the obstacles.
    \item First global search was performed with interval of 1 second and discretising the RAV to 12 points on average. The tree was expanded to depth of 5. Using iterative deepening algorithm, motion time was 3.5 seconds using $S_{ff}$.
    \item TG took 6.07 seconds using $S_{rr}$
    \item MV took 3.56 seconds using $S_{ff}$
    \item Using both MV and TG strategies, the robot took 5.31 seconds and used the maneuver $S_{fr}$. 
\end{itemize}

\section{Deficits and contributions}
\begin{itemize}
    \item There were no experiments performed on a real robot to test the theory. All the experiments that the authors show were performed in computer simulation of circular objects.
    \item There was no data given on how much actual processing time each iteration takes in TG or MV\@. Time taken was only given for global search.
    \item The velocity of the obstacles around the robot has to be explicilty provided to the robot.
    \item Only circular objects are considered. The authors defend their position by saying that any shape can be approximated by a set of circles.
    \item The advantages of this approach are
        \begin{enumerate}
            \item `It permits efficient geometric representation of potential avoidance maneuvers of the moving obstacle'
            \item No limit on number of obstacle to be avoided.
            \item No seperate representation for stationary and moving obstacle.
            \item Considers robot dynamics and actuator constraints.
        \end{enumerate}
\end{itemize}
\begin{thebibliography}{1}
    \bibitem{fiorini} Fiorini, Paolo, and Zvi Shiller. ``Motion planning in dynamic environment using velocity obstacle''. International Journal of Robotics Research, 1998.
\end{thebibliography}
\end{document}

