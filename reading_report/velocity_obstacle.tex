\documentclass[12pt]{article}
\usepackage{natbib,amsmath,amsfonts,fullpage,hyphenat,booktabs,graphicx,setspace}
\usepackage[colorlinks,linkcolor=blue,citecolor=blue,urlcolor=blue]{hyperref}
%\onehalfspacing{}
\setcitestyle{square,comma,super}
\title{Reading report \\Robot motion planning in dynamic environment \cite{fiorini} }
\author{Dharmin Bakaraniya}
\begin{document}
\maketitle{}
\begin{abstract}
This paper presents a method for computing the
motions of a robot in dynamic environments, subject to the robot dynamics and its actuator con-
straints. This method is based on the concept of
Velocity Obstacle, which defines the set of feasible robot velocities that would result in a collision
between the robot and an obstacle moving at a
given velocity. The avoidance maneuver at a specific time is thus computed by selecting robot's
velocities out of that set. A trajectory consisting
of a sequence of avoidance maneuvers at discrete
time intervals is generated by a search of a tree of
avoidance maneuvers. An exhaustive search computes near minimum-time trajectories, whereas
a heuristic search generates feasible trajectories
for on-line applications. These trajectories are
compared to the optimal trajectory computed by
a dynamic optimization that minimizes motion
time, subject to robot dynamics, its actuator limits and the state inequality constraints due to the
moving obstacles. This approach is demonstrated
for planning the trajectory of an automated vehicle in an Intelligent Vehicle Highway System sce-
nario.
\end{abstract}
\section{Summary}
\begin{itemize}
    \item Motion planning in dynamic environment is considerably more difficult that static motion planning because it needs to solve path planning and velocity planning at the same time.
    \item Motion planning in staric environment is guaranteed to come up with a solution if one exists but in dynamic environment the solution is `intractable'.
    \item The advantages of this approach are
        \begin{enumerate}
            \item `It permits efficient geometric representation of potential avoidance maneuvers of the moving obstacle'
            \item No limit on number of obstacle to be avoided.
            \item No seperate representation for stationary and moving obstacle.
            \item Considers robot dynamics and actuator constrints.
        \end{enumerate}
    \item Velocity Obstacle (VO) is an extention of Configuration Space Obstacle for a dynamic environment.
    \item For any given obstacle $B_1$ and moving object $A$, there will be a collision if $\bold{v}_{A,B_1}$ lies between the two tangents of $B_1$ ($\lambda_f$ and $\lambda_r$).
    \item The circle around which the tangents are to be drawn depend on the velocity of A. This makes a lot more sense because if A was moving very slowly then the circle around $B_1$ very small which means very specific movement will actually result into collision.
    \item The area between these tangents is called \itemit{collision cone}.
    \item Velocity obstacle is addition (Minkowski vector sum) of collision cone with the velocity $\bold{v}_{B_1}$ $$VO = CC_{A,B_1} + \bold{v}_{B_1}$$
    \item For stationary obstacle, $\bold{v}_{B_1} = 0$ and thus VO will not be translated.
    \item VO for multiple obstacle can be combined $$VO = \bigcup^{m}_{i=1}VO_i$$
    \item VO can be calculated periodically to fight against variable velocity.
    \item To avoid obstacle, the robot calculates it reachable velocities by considering the acceleration that it can achieve and the time till next interval.
    \item From this reachable velocities, the robot can then calculate reachable avoidance velocities by substracting VO from reachable velocities.
    \item To calculate avoidance trajectories, authors suggest two approaches. 
        \begin{enumerate}
            \item Global search over all feasible meaneuvers at regular interval.
            \item Heuristic search for online application when the trajectories of obstacles are not known before hand. There are 2 hueristic proposed by the authors
                \begin{enumerate}
                    \item TG (to goal): Selects highest avoidance velocity in direction of goal
                    \item MV (maximum velocity): Selects maximum velocity towards the goal within $\alpha$ variance. 
                \end{enumerate}
        \end{enumerate}
    \item The heuristic approaches is a function which prioritises survival of robot, reaching the target and minimizing travel time.
    \item TG takes 6.07 s to complete the same task (avoiding two moving obstacles with different velocities and get to other side of the road) while MV takes 4.85 s. The optimal solution takes 4.6 s for the robot to complete this task.
\end{itemize}
\section{Doubts and notes}
\begin{itemize}
    \item I could not understand most of the part in section 3.
    \item There were no experiments performed on a real robot to test the theory.
    \item There was no data given on how much actual processing time each iteration takes.
    \item The approach seems quite simple but very costly (computationally).
\end{itemize}
\begin{thebibliography}{1}
    \bibitem{fiorini} Fiorini, Paolo, and Zvi Shiller. ``Robot motion planning in dynamic environments.'' International Symposium of Robotic Research. Springer Munich, Germany, 1995.
\end{thebibliography}
\end{document}
