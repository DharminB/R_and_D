\documentclass[12pt]{article}
\usepackage{natbib,amsmath,amsfonts,fullpage,hyphenat,booktabs,graphicx,setspace}
\usepackage[colorlinks,linkcolor=blue,citecolor=blue,urlcolor=blue]{hyperref}
\setcitestyle{square,super,comma}

\title{Reading report\\High speed navigation using the global dynamic window approach\cite{brook1999high}}
\author{Dharmin Bakaraniya}
\begin{document}
\maketitle{}

\begin{abstract}
Many applications in mobile robotics require the
safe execution of a collision-free motion to a goal position. Planning approaches are well suited for achieving a goal position in known static environments,
while real-time obstacle avoidance methods allow reactive motion behavior in dynamic and unknown environments. 
This paper proposes the global dynamic window approach as a generatlization of the dynamic window approach. 
It combines methods from motion planning and real-time obstacle avoidance to result
in a framework that allows robust execution of high-velocity, goal-directed, reactive motion for a mobile
robot in unknown and dynamic environments. The
global dynamic window approach is applicable to non-holonomic and holonomic mobile robots.
\end{abstract}

\section{Summary}
\begin{itemize}
    \item Authors categorize algorithms that generate motion for mobile robots into 2 categories
        \begin{itemize}
            \item Planning algorithms
            \item Obstacle avoidance algorithms
        \end{itemize}
    \item In dynamic environment, the plans generated by planning algorithms are often inaacurate and needs replanning periodically. Obstacle avoidance algorithm on the other hand do not always produce desired motion. Thus, the author propose to combine them to reap benefits of both.
    \item The dynamic window approach\cite{fox1997dynamic} is robust in obstacle avoidance at high velocity, but it still lacks the local minima problem. That means, it can still get stuck.
    \item The authors have selected holonomic vehicles because it has an advantage over car and synchro drive type vehicles. The advantage is that it can accelerate instantaneously in all direction.
    \item Thus, the dynamic window approach has a window of rectangular shape while the holonomic dynamic window has a shape of circle. Virtually, the window in holonomic dynamic window can be infinitely large but because of computational constraints, the authors have discretised the window and used polar coordinates.
    \item In order to accelerate from a constant velocity to a desired velocity, a curve is attained in velocity space. The curvature of this curve is dependent on the amount of acceleration needed. At lower acceleration, a car like behaviour is achieved.
    \item The circular dynamic window is searched for different values of acceleration. These different acceleration result in different curves in velocity space. For each of these curves, their lengths are calculate. ``If the length of the trajectory permits the robot to
        come to a halt after moving for the duration of one
        servo tick, the motion command is considered admissible.''\cite{brook1999high}
    \item The objective function $$\Omega(p, v, a) = \alpha \cdot align(p,v) + \beta \cdot vel(v) + \gamma \cdot goal(p,v,a)$$ is used to achieve desired velocity \textit{v} and acceleration \textit{a}. Here \textit{p} is the position of the robot, $\alpha, \beta$ and $\gamma$ are parameters that can be adjusted to get desired behaviour from robot, \textit{align, vel} and \textit{goal} are function with normalised [0,1] output.
    \item $align(p,v) = 1 - |\theta|/\pi$, where $\theta$ is angle between robot's direction and goal's direction. So, it will return larger value if the robot is heading in the direction of goal.
    \item $vel(v) = \frac{||v||}{v_{\max}}$, where $v_{\max}$ is the maximum velocity that the robot can take.
    \item $goal(p,v,a) = 1$ if the trajectory passes through goal region otherwise it is 0.
    \item The authors extend the dynamic window approach\cite{fox1997dynamic} and holonomic dynamic window approach with a global planning algorithm which is very efficient (can be executed each servo tick).
    \item For connectivity information, a model of the environment can be provided but it is not needed. The sensory information can be easily used to create a model. The information integration can have errors as well but these errors will not inhibit the performance as the accurate sensory information is updated very frequently. 
    \item The NF1 algorithm (wavefront explansion from goal) is used as global navigation planner for connectivity information. A common problem with NF1 algorithm is that it grazes the robot with obstacles but when combined with dynamic window approach, this problem gets solved.
    \item Tradition NF1 algorithm is applied only once on the whole map because the goal and obstacles remain static. But here NF1 is called everytime a new motion command is needed. Also, it is not applied on the whole map, but instead, it is applied on a rectangular area aligned towards goal with dynamic width. The width is increased until the goal is connected to current position.
    \item In order to incorporate NF1 in existing holonomic dynamic window objective function, \textit{align} function is replaced with \textit{nf1} function and a new function $\Delta nf1$ is added.$$\Omega_g(p, v, a) = \alpha \cdot nf1(p,v) + \beta \cdot vel(v) + \gamma \cdot goal(p,v,a) + \delta \cdot \Delta nf1(p,v,a)$$
    \item \textit{nf1} function returns higher value if the robot is aligned with the gradient of NF1. The value of this function can be determined by examining the value of all the neighbours lying at a certain distance from robot's current position.
    \item $\Delta nf1$ function  ``indicates by how much a
        motion command is expected to reduce the value of
        the NF1 during the next servo tick.''\cite{brook1999high}
    \item The experiments were performed on Nomad XR4000 with maximum velocity of 1.2 m/s and maximum acceleration of 1.5 m/s$^2$ with SICK laser with range 180$^0$ and accuracy of 1cm. The servo rate of 15Hz was achieved for a map of size 30m$\times$30m with resolution of 5cm. The robot was able to navigate reliably at high velocity (1 m/s) in 2 different maps of size 6m$\times$6m and 10m$\times$10m. No prior knowledge was provided to the robot about the environment.
\end{itemize}

\section{Deficits and contributions}
\begin{itemize}
    \item The experiment does not involve moving obstacles or moving goal.
    \item The authors only address holonomic vehicles.
    \item The problem of local minima for approaches like dynamic window approach\cite{fox1997dynamic} is solved.
    \item The algorithm is quite fast and efficient.
\end{itemize}
\begin{thebibliography}{1}
	\bibitem{brook1999high} Oliver Brock, Oussama Khatib, ``High speed navigation using the global dynamic window approach'', \textit{International Conference of Robotics and Automation}, 1999
    \bibitem{fox1997dynamic} Dieter Fox, Wolfram Burgard, and Sebastian Thrun, ``The dynamic window approach to collision avoidance'', \textit{IEEE Robotics and Automation Magazine} 1997
\end{thebibliography}


\end{document}
