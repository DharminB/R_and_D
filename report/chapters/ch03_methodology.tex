%!TEX root = ../report.tex
\chapter{Methodology}

\section{Setup}
\begin{itemize}
    \item A planar mobile robot with a kinematic model is spawned at initial position. 
    \item The robot needs to travel to the goal position. 
    \item The robot should avoid collision with static and moving obstacles. 
    \item A collision is considered to happen when the physical body (model in case of simulation)
        is in contact with physical body of the obstacle.
    \item The time duration taken for the robot to travel from intial to goal position is measured.
    \item The position of the robot in the environment is available to the motion planner.
    \item A global planner is also available to the motion planner if it needs one.
    \item The position, velocity and shape of the moving obstacle is also available to the motion
        planner.
    \item The experiment is performed multiple times and each time following things are measured
        \begin{itemize}
            \item time to reach goal
            \item number of collisions
            \item number of times planner failed to plan
        \end{itemize}
\end{itemize}

\section{Experimental Design}
\begin{itemize}
    \item KUKA youbot\cite{bischoff2011kuka} is used which uses ROS\cite{quigley2009ros}. 
    \item A gazebo simulator\cite{koenig2004design} is used for simulating the desired environment.
    \item Global planner is used from ROS navigation\cite{rosnavigation} if required. 
    \item All obstacles are cyclinders of diameter 30 cm and height 50 cm. None of the obstales move 
        faster than 0.4 m/s on average. All obstacles follow to and fro motion. The obstacles have 
        linear acceleration and deceleration. This implies that their velocity is constantly changing.
\end{itemize}

\newpage{}
\subsection{Test case 1: Single obstacle single room}%
\label{sub:test_case_1_single_obstacle_single_room}
    The goal is in line of sight of the robot. The motion planner does not need to rely on 
    global planner. A single obstacle is moving between the robot and the goal as shown in 
    Figure~\ref{fig:single_obs_single_room}. The size of the room is 4 meters in length and 3 
    meters in width. This represents a small room or a small subsection of a large room in a 
    typical indoor environment. Apart from walls of the room, no additional static obstacles are
    simulated. The maximum opeining for the robot is 2.7 meters when the obstacles is at the 
    end points of its path.
\begin{figure}[htpb]
    \centering
    \includegraphics[width=0.8\linewidth]{name.ext}
    \caption{\label{fig:single_obs_single_room}}
\end{figure}

\newpage{}
\subsection{Test case 2: Two obstacle single room}%
\label{sub:test_case_2_two_obstacle_single_room}
    The goal is still in line of sight of the robot. This test case is a logical extension of
    the previous test case. There are 2 obstacles between the robot and the goal. The difficulty
    is almost double because the opening for the robot is always less than 1.7 meters. This is 
    because the obstacles are offset by 50\% as shown in Figure~\ref{fig:two_obstacle_single_room}.
\begin{figure}[htpb]
    \centering
    \includegraphics[width=0.8\linewidth]{name.ext}
    \caption{\label{fig:two_obstacle_single_room}}
\end{figure}

\newpage{}
\subsection{Test case 3: Double room}%
\label{sub:test_case_3_double_room}
    Unlike the previous test cases, the goal is not in line of sight in this test case. The 
    robot has to exit its current room, travel through a corridor filled with moving obstacles 
    and enter another room to get to the goal position. The corridor is 2 meters wide and 8 meters
    in length. It contains 6 obstacles which move in 3 tracks as shown in Figure~\ref{fig:double_room}.
\begin{figure}[htpb]
    \centering
    \includegraphics[width=0.8\linewidth]{name.ext}
    \caption{\label{fig:double_room}}
\end{figure}




