%!TEX root = ../report.tex

\chapter{Introduction}

Motion planning in dynamic environment for planar mobile robots has a lot more application compared to one in static 
environment as the robot can be deployed in real environment among humans and other robots. 
The robot does not need to be in its own special and controlled environment. The robot can be 
used in day to day life of an average person (for example, automated taxi or public transport).
A mobile robot which can navigate in dynamic environment is extremely useful for industries to 
automate their production line without changing their whole infrastructure.
This need is increasing with the availability of cheap and easily accessible robots. 

\section{Motivation}
\label{sec:motivation}
By comparing different approaches of motion planning in dynamic environment on same constraints and parameters, 
a well informed decision regarding the best approach for a given application is achieved. By finding out the 
current best approach and its limitations, we can also start focusing on solving those limitations
rather than finding solutions for solved problems. 
By solving problem of motion planning in dynamic environment, we can ensure 
\begin{itemize}
    \item Safe environment for humans and for robots
    \item Cost effective transportation of goods (and people\cite{prassler2001robotics})
\end{itemize}


\section{Challenges and Difficulties}
\label{sec:challenges_and_difficulties}
Robot motion planning can be considered solved for static environment\cite{large2005navigation}.  
However, the robot does not have perfect information about the surrounding environment in a
dynamic environment. The task to calculate a motion control for such situation is considered
a NP-hard problem\cite{keshmiri2009overview}. 

Simple put, motion planner needs to generate a plan that
enables the robot to reach the desired target position while avoiding static and moving obstacle in minimum time.
Additionally, the planner should satisfy kinematic and dynamic constraints of the robot.
The planner needs to be fast enough that the robot can react to its environment in real time.
It is desirable that these tasks are performed in \textit{efficient} manner. By efficient, we mean 
the planner does not require lot of sensor data or lots of computational resources.

The state of the art approaches test their methods and algorithm with different parameters such as
    \begin{itemize}
        \item kinematics of robot (holonomic, differential drive or car-type)
        \item availability of map
        \item shape of moving obstacles (circular, convex polygon, etc)
        \item kind of motion of moving obstacles (same direction with constant velocity, varying
            direction with constant velocity, varying direction with varying velocity)
        \item accuracy of localisation of robot
        \item amount of information available to robot about the obstacles (shape, position and velocity)
    \end{itemize}
This makes the task of determining the better approaches from the rest difficult and the 
    result may be unreliable.



\section{Problem Statement}
\label{sec:problem_statement}
To compare motion planners and their approaches as a whole, a literature comparison is necessary.
With this initial comparison, many of the approaches could be eliminated because of one or more of
the following reasons
\begin{itemize}
    \item The approach or planner does not work with certain parameter that is desired. For example, 
        if an approach is only applicable to differential drive but we want the motion planner to 
        navigate a car. Another example would be that an approach needs extreme amount of computation
        to navigate but we want use the planner on a cost effective and small mobile robot.
    \item The approach is highly criticised by peer reviewers
\end{itemize} 
After eliminating all the approaches which are not applicable, we still have a number of approaches
which are tested on different robot under different assumptions (as mentioned in~\ref{sec:challenges_and_difficulties}).
A comparative evaluation of the implementation of these approaches under specific assumption is 
needed to find out the best approach. To compare the performance of the tested algorithm, we choose
travel time, number of re-plan attempts and number of collisions as measurements for a better motion
planner for dynamic environment.

