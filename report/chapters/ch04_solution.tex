%!TEX root = ../report.tex

\chapter{Solution}

\section{Implementation details}
\label{sec:implementation_details}

\subsection{Moving obstacles}%
\label{sub:moving_obstacles}

\begin{itemize}
    \item The moving obstacles are animated with the help of a gazebo plugin modified from\cite{animatedBox}.
    \item The plugin takes information like key frames and position and orientation of object at each
        key frame and animates the objects in gazebo. It also advertises the position and orientation
        of the obstacles at each iteration to a gazebo channel. This is then read by a ROS node and 
        published as a ROS message every 15 ms.
    \item Additionally we have created small modular ROS nodes which 
        \begin{itemize}
            \item calculate relative position of obstacles w.r.t.\ robot
            \item calculate instantaneous velocity of obstacles
            \item publish robot's position in the world
            \item converting moving obstacle information to costmap (Refer~\ref{sub:obstacle_trajectory_to_costmap})
        \end{itemize} 
        to be used if a planner needs it.
    \item The source code for these plugins and helper nodes are available at~\cite{movingObstacleGazebo}
\end{itemize}

\subsection{Obstacle trajectory to costmap}%
\label{sub:obstacle_trajectory_to_costmap}
    This ROS node script (\texttt{obs\_to\_costmap}) calculates the instantaneous velocity of the 
    moving obstacle and creates fake occupancy information for the future track of the moving obstacle. 
    This tricks a planner into thinking certain areas are occupied whereas, in reality, they are not. 
    This node can be used with any static local planner which considers costmap\cite{costmap} information.
    
    We find that choosing appropriate time duration for extrapolating future trajectory is a trade
    off. If the time is too short, the planner might collide with the obstacle. On the other
    hand, if the time is too long, the planner's path may be blocked completely and it will be forced
    to wait until the costmap clears (this will increase the time to reach goal). 

\section{Tested algorithm}
\subsection{TEB local planner}%
\label{sub:teb_local_planner}

\begin{itemize}
    \item This planner\cite{tebLocalPlanner} is implemenatation of Timed elastic band 
        approach\cite{rosmann2015planning}.
    \item The planner is used with ROS navigation~\cite{rosnavigation}. It is used as local
        planner in \texttt{move\_base} which publishes \texttt{cmd\_vel} message to control robot. 
        The planner makes small changes to the path generated by the global planner.
    \item To get the information about the dynamic obstacles, teb local planner uses \texttt{obstacles} 
        ROS message. It can contain the information about the shape of the robot (as polygon), its
        position and its velocity. This information is provided to teb local planner by one of the
        helper nodes in~\cite{movingObstacleGazebo}.
    \item \texttt{move\_base} listens to ROS topic \texttt{move\_base\_simple/goal} to get the position
        and orientation of the goal.
\end{itemize}

\subsection{Spline based planner}%
\label{sub:spline_based_planner}
\begin{itemize}
    \item This planner\cite{omgtools} is implementation of Spline based planner approach\cite{mercy2017spline}.
    \item By default the planner works in its own environment and simulator. The planner can be 
        incorporated with ROS framework\cite{p3dxMotionplanner}. We have, however, not made it 
        compatible with ROS navigation.
    \item Eventhough the planner is incorporated with ROS, it does not use gazebo simulator. The 
        plans in its own simulator and the wrapper built around this code are used to translate the 
        information from one simulation to another. This could be a mojor handicap for this planner 
        because it uses 2 simulation simultaneously which will slow down the simulation and increase
        the response time.
    \item The planner simulator requires the size of rooms and position and shape of obstacles before
        starting. This may become a problem if the robot does not have information about each moving
        obstacle. For the purpose of this experiment, we provide the planner with information about
        all moving obstacle no matter if the robot can sense the obstacle with its sensors or not.
        % We observe that the gazebo simulation runs at 40 to 50\% speed of real time
        % when this planner is used whereas ROS navigation\cite{rosnavigation} planner enables gazebo
        % to run at 80 to 90\% speed of real time.
\end{itemize}

\subsection{DWA local planner}%
\label{sub:dynamic_window_approach_planner}
\begin{itemize}
    \item DWA local planner\cite{dwa} is implementation of Dynamic window approach~\cite{fox1997dynamic}.
    \item This planner works as a local planner in ROS navigation\cite{rosnavigation}. The DWA approach
        alone cannot work without the global planner because it only consider a limited area around
        the robot to plan a local path. The planner needs to be configured to make changes to the
        global plan upto a certain amount.
    \item By default it is used for static motion planning. This puts DWA planner at a major disadvantage
        because it cannot predict the position of moving obstacle in future. 
    \item For this reason, we couple the planner with a \texttt{obs\_to\_costmap} which changes the costmap for the 
        environment according to the velocity of the moving obstacles.
\end{itemize}

\subsection{EBand local planner}%
\label{sub:eband_local_planner}
\begin{itemize}
    \item This planner\cite{eband} is implementation of Elasic band approach\cite{quinlan1993elastic}.
    \item Similar to DWA planner and teb planner, EBand local planner also works as a local planner
        in ROS navigation\cite{rosnavigation}. This is because it makes changes to the path generated
        by the global planner. 
    \item The planner gets the information of the obstacles from costmap and does not use any other 
        source of information.
\end{itemize}

\subsection{EBand with obs\_to\_costmap}%
\label{sub:eband_with_obs_to_costmap}
\begin{itemize}
    \item Similar to DWA local planner, we couple EBand planner\cite{eband} with \texttt{obs\_to\_costmap}.
    \item As the planner only considers costmap information, the new costmap will provide information
        about the future poses of the moving obstacles to the planner. We expect this approach to 
        perform better than Eband (\ref{sub:eband_local_planner}) in terms of number of collisions and worse in
        terms of time taken to reach goal. We expect that because this approach will be forced to 
        choose safer trajectory which might force the robot to wait for the path to clear or force 
        the robot to go around the moving obstacle rather than passing it.
\end{itemize}







