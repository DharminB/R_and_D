%!TEX root = ../report.tex

\chapter{Conclusions}

\section{Contributions}
\label{sec:contributions}
We compared some of the state of the art approaches in motion planning in dynamic environment.
We setup a experiment and test criteria to compare motion planners in dynamic environment.
We compare 5 motion planners in 3 test cases with increasing complexity in gazebo simulator.
We analysed a better motion planner out of the 5 based on the test criteria set earlier.
We provide software for easily modifying a ROS navigation code base for static environment to
one for static and dynamic environment.

\section{Lessons learned}
\label{sec:lessons_learned}
A motion planner which uses information about moving obstacles directly performs better
than a motion planner which learn that information indirectly. 
A mathematical simulation with circular robots, circular obstacles and robot fed with
all true values along with perfect controller tests motion planning
algorithm less robustly than a simulation which uses real robot models. 
A motion planner written in python programming language is much slower than a motion planner
binary compiled from C++ source code. We observe that the gazebo simulation runs at 40 to 50\% speed of real time
when spline based planner\cite{omgtools} planner is used whereas ROS navigation
planners\cite{dwa, tebLocalPlanner, eband}  enables gazebo
to run at 70 to 80\% speed of real time.

\section{Future work}
\label{sec:future_work}
Using perception techniques to gather information about the moving obstacles rather than
gaining it from simulator would provide a comparison for more realistic world.
Comparing motion planners on real robot with humans as moving obstacles would be the next step to 
compare motion planners.
Comparing planners for different robot models with different kinematic and dynamic constraints would
give a better idea about a more general and robust motion planner.
Trying variable amount of \texttt{lookahead\_time} for \texttt{obs\_to\_costmap} for static motion planners
and comparing them with dynamic motion planner.
Finally, combining and creating a motion planner by learning from the results of the comparison, might
provide best results.
