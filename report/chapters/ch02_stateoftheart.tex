%!TEX root = ../report.tex

\chapter{State of the Art}

\section{Related work}%
\label{sec:related_work}

Mohanan et al.\cite{mohanan2018a} covers 101 research papers that were published between 1985 and 
2016 in the field of motion planning in dynamic environment. This survey is very close to the 
current work in terms of publication date. They have considered most of the work in this field and 
presented it in different sections on the basis of school of thought. Even though this work covers 
many approaches, the comparison is lacking. The only comparison is within a school of thought and 
that too is quite shallow. Additionally, the authors do not provide a valid reasoning behind any 
of their claims and neither does their work contain any experimental study. 

Hoy et al.\cite{hoy2015algorithms} provides a survey for algorithms which provide collision free 
navigation for robots. This work is fairly detailed. However, this work cannot be considered very close to 
our work because it mainly covers obstacle avoidance comparison, but it does contain a section for 
dynamic obstacle avoidance. They have given quite a lot of importance of collision avoidance using 
boundary following. Even though it provides a comparison between main approaches based on numerous 
criterion, it still does not evaluate these approaches on same parameter. They have provided their 
comparison solely based on the information provided by the original research papers.

Keshmiri et al.\cite{keshmiri2009overview} provides a survey specifically for motion planning 
in dynamic environment. It covers all the approaches presented in research papers published between 
1986 and 2008 totalling up to 150 papers. They have provided a comparison on how much contribution 
has been made in motion planning field based on different approach but regarding the actual 
approaches itself, only a summary of at most 2\-3 sentences for each approach is provided. The authors
have not performed any test on the approaches. This work was meant for just providing a survey 
for contribution in different school of thought.

Surveys~\cite{fujimura1991motion} and~\cite{tsubouchi1996motion} are quite dated and does not 
cover many state of the art approaches in motion planning for dynamic environment.

Existing approaches generally provide critique and deficiencies on their previous works. These are 
generally helpful for the reader at the time of publishing but they can only provide a critique on 
the approaches that were published before them. Additionally, their critique are generally about 
the problem that they have solved in their work. Therefore, though these comparisons are helpful 
at that time for the reader to better understand the presented work, they might not be completely 
useful for an unbiased comparison.

\section{Motion planning approaches}%
\label{sec:motion_planning_approaches}

\subsection{Velocity based}%
\label{sub:velocity_based}

\subsubsection{Dynamic window}%
\label{subsub:dynamic_window}

Fox et al.\cite{fox1997dynamic} proposed the original idea which was simply optimizing a function which balances robot's distance from goal ($heading$), distance from nearest obstacle ($dist$) and current velocity ($vel$).  $$ G(v,w)= \sigma(\alpha \cdot  heading(v,w) +\beta \cdot dist(v,w) + \gamma \cdot vel(v,w)) $$
    Here $\alpha$, $\beta$, $\gamma$ and $\sigma$ are constant that can be tuned according to need.
This is a reactive approach, which means that the algorithms only thinks about the next iteration.
It takes care of the kinematic and dynamic constraints by converting the current sensor readings in velocity space.
Velocity space is a 2D space where X axis is angular velocity of the robot and Y axis is linear velocity.
The robot is represented as a dot in this velocity space. The controls are bounded by the rectangle around this dot which represents the dynamic constraints of the robot.
This approach, despite being robust, simple and fast did not work for dynamic environment. This method also had local minima problem.

Later, Brock et al.\cite{brock1999high} extended this approach for global path planning and for dynamic environments by combining it with NF1 algorithm. This eradicated the problem of local minima. 
The experiments did not contain dynamic objects.
This approach is later extended by~\cite{ogren2005convergent} where they provide a version of DWA which is convergent but they only consider static environment.

The current best extension of dynamic window approach (time varying dynamic window (TVDW)) was proposed by~\cite{seder2007dynamic}. This approach provides very robust obstacle avoidance method for dynamic environment.
The authors have combined FD* path planning algorithm with dynamic window approach.
This method uses moving cells for representing obstacles. This relieves the method from considering the shape of the obstacle.
The obstacles are represented in the grid map at the location where the planner thinks they will collide. This helps the algorithm to calculate alternate trajectories efficiently.

Another extension was proposed by~\cite{chung2009safe}, where they emphasise on the safety of the robot. They create a map with collision distance ($d_{collision}$) which takes into account any obstacles that are outside the field of view or occluded by other static objects. This approach provides a trajectories with longer travel time but the overall safety of the robot is far more compared to classical DW approach.

\subsubsection{Velocity obstacle}%
\label{subsub:velocity_obstacle}

Originally developed by Fiorini et al.\cite{fiorini1998motion}, this approach proposes to avoid obstacles by choosing velocity outside \textit{collision cone}.
The robot and the obstacles are considered circular. The robot contains the information about the position, orientation and the velocity of the obstacle.
The obstacles are believed to travel at constant velocity and same orientation.
The set of relative velocity of the robot and obstacle which result in collision is termed as a collision cone. 
By selecting an appropriate velocity outside the collision cone (avoidance velocity) of all the moving and static obstacles, the robot can ensure collision avoidance. Additionally, the choice of admissible velocity is restricted to the velocity allowed by the kinematic and dynamic constraints. These are called \textit{admissible avoidance velocities}.
This approach unifies the representation for avoiding static and dynamic obstacles.
This idea has since been transformed to incorporate many scenarios~\cite{prassler2001robotics}.

One of the extensions, NLVO, was proposed by Shiller et.\ al.~\cite{shiller2001motion}. They extend the classical velocity obstacle approach that deals with obstacles moving linearly with constant velocity to propose \textit{non linear velocity obstacles} that can deal with obstacles moving on arbitrary path at constant velocity.
The concept of NLVO was used very successfully in~\cite{large2005navigation}, where they successfully implemented and test their extension of NLVO on a car-like robot. They have adapted the method such that it can deal with non circular robot and obstacles. As this can be computationally expensive, they used graphics library to speed up the collision calculation.

A reactive method for VO was developed in~\cite{belkhouche2009reactive}. They propose the idea of virtual plane which is obtained by transforming moving obstacles into static obstacles and transforming robot to a virtual robot. This plane is used to calculate the time to collision and collision position. By using this information, even a simple planner can be used to avoid collision. The authors state that any of the previous approaches that were used to solve static motion planning can be used here for desired outcome. They implemented and test their approach with multiple moving obstacles with acceleration.

For multi robot systems, Van den berg et al.\cite{van2008reciprocal}\cite{van2011reciprocal} have extended VO approach. The agents calculates the collision cone for another agents that are on the course of collision with it and chooses an admissible avoidance velocity that makes the changes only half of what is required. As all the agents are running the same algorithm and both of them will make half the effort, they will avoid collision without communicating with each other.

\subsubsection{Velocity space}%
\label{subsub:velocity_space}
An approach using velocity time space was proposed by~\cite{owen2005motion}\cite{owen2006a}. If the obstacles move with constant velocity in same direction then the collision time and distance can be calculated. This information is mapped to velocity time space. This creates a surface in this 3D space which is used to choose a velocity for the robot so that it does not collide with this obstacle. The target is also mapped to velocity space.This space was named Dynamic velocity space (DVS). Simulated experiments were performed with 3 dynamic obstacles.



\subsection{ICS based}%
\label{sub:ics_based}

``Inevitable collision states (ICS) are states that will end up in a collision, no matter which trajectory is followed by the robot''\cite{fraichard2004inevitable}. This formal definition and a few important ground work was provided in~\cite{fraichard2004inevitable}. This works on a state time space concept. Later,~\cite{fraichard2007short} declared that ICS is compliant with the safety criteria necessary for a robot to move safely in a dynamic environment. 
These safety criteria are as follows
    \begin{enumerate}
        \item ``The robot should consider its own dynamics
        \item The robot should consider the future behaviour of environment objects
        \item The robot system should reason over an infinite time horizon''\cite{fraichard2007short}
    \end{enumerate}
A partial motion planner combined with ICS-checker was proposed in~\cite{petti2005safe}. They provide an additional property which proves that if the trajectory is collision free and the end state is not an ICS, then the whole trajectory is ICS-free. This allows them to calculate partial path and only check the final state for ICS\@. Their collision checker is rapidly exploring random tree based and only checks for braking trajectories of the robot. The algorithm was implemented on a car-like robot in simulation.

Later~\cite{martinez2009collision} proposes ICS-AVOID, an ICS based collision avoidance approach. They use the ICS-checker proposed by~\cite{martinez2008efficient} and combine it with \textit{safe control kernel}. This guarantees that the sampling output will always contain at least one safe trajectory. This enables their algorithm to go from one ICS-free state to another without failure. This is later compared with state of the art approaches based on dynamic window (TVDW\cite{seder2007dynamic}) and velocity obstacle (NLVO\cite{large2005navigation}). These experiments were performed in simulation and the information about the future trajectories of the obstacles (up to a limited time horizon) were provided to each of the algorithm. The results show that when same amount of information is provided, ICS-AVOID performs far better than TVDW and quite better than NLVO\@.

An on-line planner based on the concept of NLVO and ICS was proposed by~\cite{gal2009efficient}\cite{shiller2010nonlinear}. It is based on selecting an appropriate time horizon such that the robot will either pass ahead of the obstacle or stop before the obstacle. The NLVO till that carefully calculated time horizon guarantees that the robot never ends up in an ICS state. The approach was implemented on a robot in simulation and it was tested with 70 moving obstacles. 

The ICS method was extended to probabilistic domain by~\cite{althoff2010probabilistic}. This method was later extended by the same authors in~\cite{althoff2012safety}. They introduce probabilistic collision state (PCS). So, instead of checking for collision, they are calculating the probability for collision. This probability is simply the minimum probability of a collision occurring by following a trajectory out of all the possible trajectory. This number determines the probability of that state being an ICS\@. The probabilities are calculated by considering the motion model of the obstacles and sampling from the probability density function of their resultant position at the end of some time interval. They have used only braking trajectories for collision checking since the infinite time horizon is infeasible. The authors have also considered collisions that occur among the obstacles excluding the robot. The authors have performed a string of experiments to check the effects of different parameters on PCS collision avoidance system on a car-like system.

Since guaranteed ICS free navigation is impossible, Braking ICS was proposed\cite{bouraine2012provably}. The authors state that Braking ICS free navigation is the next best solution. Their algorithm PASSAVOID is \textit{provably passively safe} meaning that the robot will be at rest if a collision occurs. A braking ICS is a state such that no matter what braking trajectory is executed by the robot, it will still collide. They provide this algorithm under the assumption of limited field of view in an unknown dynamic environment. Simulated experiments were performed for a car like robot in the middle of 22 arbitrarily moving obstacles.

An extension of PCS based navigation was developed by~\cite{hernandez2015application}. The robot constantly updates a probabilistic collision matrix based on the model of the environment. For each obstacle, sampling is done based on the previous model which considers its velocity, acceleration, direction of movement, etc. For trajectory calculation, authors have used time dependent Dijkstra algorithm. In exchange for a little higher collision probability, the authors get a huge computational speedup along with a shorter trajectory length by setting a minimum probability restriction. This means that if the collision probability of a cell is lower than a predefined threshold, then that cell is considered safe to traverse. The authors have performed extensive simulation experiments with trajectory generation with different parameters in order to explain the algorithm. For online planning, the authors have modified their algorithm so that the robot only follows the planned trajectory for certain period of time and alters the plan based on the newly available sensor data. The algorithm was tested on a real robot and pedestrians were used as obstacles. The robot was able to get from one position to another with very less collision probability.

\subsection{Roadmap based}%
\label{sub:roadmap_based}

Probabilistic roadmap approach was proposed by~\cite{hsu2002randomized}. ``A PRM planner samples the robot’s configuration space at random and retains the collision-free samples as milestones.  It then tries to connect pairs of milestones with paths of pre-defined shape (typically straight-line segments in configuration space) and retains the collision-free connections as local paths. The result is an undirected graph, called a probabilistic roadmap, whose nodes are the milestones and the edges are the local paths''\cite{hsu2002randomized}. This is a state space based approach. The sampling is done while considering kinematic and dynamic constraints imposed by the robot. The authors proved that as the number of sample increases, the probability of finding a path increases exponentially. Their approach is mainly for path planning, but they have provided with some additional techniques like sensing error counter measure, trajectory optimization, safe mode planning and on-the-fly replanning which makes their approach quite suitable for dynamic environment. They performed experiments to prove that a robot with any shape, kinematic and dynamic constraint was able to traverse. Additionally they also experimented with moving obstacle with random trajectory.

An approach\cite{van2005roadmap} was proposed for motion planning in dynamic environment where the global roadmap for static obstacles is already calculated. The authors provide a local and a global planning method which uses the previously given roadmap and only considers the dynamic obstacles. The problem is thus reduced to 1D state space. The local planner only considers going from one vertex to another in the roadmap, while the global planner employs multiple \textit{probes} that individually run the local planner algorithm to find the global path. Experiments were performed for a 6 DOF robot in a 3D space in a simulation.

In~\cite{van2006anytime}, the authors propose a method to calculate and recalculate path to the goal in dynamic environment. This approach is similar to the extension from~\cite{hsu2002randomized}. The method creates an anytime plan from the current position to the goal position which is constantly updated based on the sensor information. Anytime D* algorithm is used to calculate the trajectory in state time space. Experiments were performed for a point sized robot in an simulated dynamic environment. The moving obstacles can change their direction randomly but this information is not available to the robot. The robot still only collides approx. 20\% of the time.

\subsection{Elastic Band}%
\label{sub:elastic_band}
Elastic bands approach was proposed in~\cite{quinlan1993elastic} in order to ``close 
gap between global path planning and real time sensor based robot control''\cite{quinlan1993elastic}.
The authors proposed an intermediate step between global planner and controller which deforms the
path produced by global planner based on local observations and feeds it to controller. This 
deformation is done by applying 2 types of forces (internal contraction and external repulsion)
which makes the resultant path shortest. The elastic band is represented using a set bubbles with 
certain position and radius. This set is then manipulated at run time by changing the position
and radius of the bubbles and adding or removing of bubbles. The authors have provided simulated 
experiments with multiple static and single moving obstacle with a holonomic robot.

Timed elastic band trajectory planning approach~\cite{rosmann2013efficient} proposes an 
approach to make adjustments to the plan generated by global planner. Here, the authors 
mention that they only deal with local trajectory and not the complete path when optimising.
They mention that this makes the structure of optimisation problem sparse allowing them real
time feedback. They explain timed elastic band as an extension of elastic band. Each configuration
of elastic band is accompanied with a time duration. This duration represents the time that
the robot will take to go from previous configuration to the next. The parameters such as
velocity, acceleration, clearance from obstacles and compliance with robot's kinematic and
dynamic constraints depend mostly on 2--3 adjacent configurations. The authors use hyper-graph 
to represent the constraints of this sparse system. This optimization occurs in control loop
which enables the robot to get the motion commands directly. The authors preformed experiments 
with a simulated and a real differential drive robot in presence of an moving obstacle. They 
also tested 4 static obstacles. They report that each iteration takes on average 48 ms. This 
approach has a possibility for local minima problem because the planner only takes into account
few future configuration. 

The local minima problem mentioned above is addressed in~\cite{rosmann2015planning} employing
the concept of homotopic trajectories (trajectories with same initial and final position which
can be continuously morphed into one another). The robot continuously tries to find homotopic 
trajectories around the obstacle and chooses the one which satisfies the constraints the best.
The authors compare their approach with DWA and their previous approach\cite{rosmann2013efficient}
on a differential drive robot with 8 moving obstacles. They show that their approach is able to 
reach the goal at maximum velocity.

\subsection{Spline based}%
\label{sub:spline_based}

An approach proposed in~\cite{mercy2016real} extends~\cite{loock2015b} for real time motion
planning for robots in dynamic environment. ``Splines are piecewise polynomial functions 
that can easily represent complex trajectories, using few variables''\cite{mercy2016real}. 
The approach in~\cite{loock2015b} provides motion planning in dynamic environment. The 
authors of~\cite{mercy2016real} extend that approach by adding moving obstacles to the set
of constraints (adding hyperplanes which separate obstacles position and robot position). 
They add soft constraints for distance from nearest obstacles for extra safety. Additionally
they add receding horizon control and suggest to use high rate of observation and control 
loop for better execution. They perform experiments on simulated and real holonomic robot
with static and moving obstacles. They also consider the possibility of an obstacle moving
faster than the highest velocity of robot. 

The above approach is further extended for robot with bicycle kinematics in~\cite{mercy2017spline}
and a more detailed mathematical explanation is provided. They conduct experiments with 
holonomic, unicycle and bicycle model of robots in dynamic environment in simulation. 
Additionally they conduct experiments on real holonomic and differential drive robot with 
single moving obstacle. The average time taken for each iteration is twice in case of differential
drive experiment with a maximum of 0.98 seconds.


\subsection{Other approach}%
\label{sub:other_approach}

\subsubsection{Nearness diagram}%
\label{ssub:nearness_diagram}
A reactive navigation approach was proposed in~\cite{minguez2004nearness}.
They employ \textit{situated activity} paradigm. The robot chooses from a set of 5 behaviour
based on the sensor information. The sensor information is like vector field histogram.
The robot behaves differently when there are object on its sides compared to where it is 
directly aligned to the goal. This algorithm was proved to not have local minima. The 
authors state that the proposed method only has one parameter to be selected compared to 
other approaches where numerous parameters determine the behavior of the robot. Experiments 
were performed on a real holonomic robot in a cluttered environment.


\subsubsection{Potential field}%
\label{subsub:potential_field}
Concept of potential field is quite dated. However a potential field approach for dynamic 
environment was proposed in~\cite{ge2002dynamic}. 
Here the target and obstacles are moving and they generate attractive and repulsive forces 
respectively. Problem of local minima was addressed. Calculations for holonomic robot and 
differential drive robot were provided. Experiments were performed in simulation and on real robots. 

\subsubsection{Biologically inspired}%
\label{subsub:biologically_inspired}
A biologically inspired approach for differential drive robots was presented 
in~\cite{savkin2013simple}. Here the robot calculate the angle made by the obstacle in the 
field of sensing and calculates the linear and angular velocity that it must obey in order 
to avoid the obstacle. The robot follows two type of motion: avoidance motion and straight 
to goal motion. The robot switches between these modes to reach the target location. 
Experiments with multiple moving obstacles were performed in real and simulated environment. 
The approach was compared with velocity obstacle approach and the proposed method was faster 
to reach the target location.


\section{Comparison}%
\label{sec:comparision}

\begin{table}[htpb]
    \centering
    \begin{tabular}{cccccc}\toprule
        \textbf{Approach} & \textbf{Method} & \textbf{Vehicle} & \textbf{Obstacle} & \textbf{Obstacle} & \textbf{Experiment} \\
                          &    & \textbf{type} & \textbf{restriction} & \textbf{shape} & \\\toprule
        \multirow{5}{*}{DWA}&\cite{fox1997dynamic}              & Holonomic     & \-- & All & real \\
                            &\cite{brock1999high}               & Holonomic     & \-- & All & real \\
                            &\cite{ogren2005convergent}         & Holonomic     & \-- & \-- & sim \\
                            &\cite{seder2007dynamic}            & Holonomic     & cv, vd & All & sim \\
                            &\cite{chung2009safe}               & Holonomic     & cv, cd & All & real \\\midrule
        \multirow{6}{*}{VO} &\cite{fiorini1998motion}           & Holonomic+    & cv, cd & circular & sim \\
                            &\cite{prassler2001robotics}        & Unicycle+     & cv, cd & All & real \\
                            &\cite{shiller2001motion}           & Holonomic+    & cv, vd & circular & sim \\
                            &\cite{large2005navigation}         & Bicycle       & cv, vd & All & sim \\
                            &\cite{belkhouche2009reactive}      & Unicycle      & vv, vd & circular & sim \\
                            &\cite{van2008reciprocal}           & Holonomic     & vv, vd & circular & sim \\
                            &\cite{van2011reciprocal}           & Holonomic     & vv, vd & circular & sim \\\midrule
        \multirow{1}{*}{VS} &\cite{owen2005motion,owen2006a}    & Unicycle+     & cv, cd & All & sim \\\midrule
        \multirow{9}{*}{ICS}&\cite{petti2005safe}               & Bicycle+      & vv, vd & All & sim \\
                            &\cite{martinez2009collision}       & Holonomic+    & cv, vd & circular & sim \\
                            &\cite{martinez2008efficient}       & Bicycle+      & cv, vd & circular & sim \\
                            &\cite{gal2009efficient}            & Holonomic+    & cv, vd & circular & sim \\
                            &\cite{shiller2010nonlinear}        & Holonomic+    & cv, vd & circular & sim \\
                            &\cite{althoff2012safety}           & Bicycle+      & vv, vd & circular & sim \\
                            &\cite{bouraine2012provably}        & Bicycle+      & vv, vd & All & sim \\
                            &\cite{hernandez2015application}    & Unicycle+     & vv, vd & All & sim, real \\\midrule
        \multirow{3}{*}{Roadmap} &\cite{hsu2002randomized}      & All           & vv, vd & All & sim, real \\
                            &\cite{van2005roadmap}              & All           & \-- & circular & sim \\
                            &\cite{van2006anytime}              & All           & cv, vd & circular & sim \\\midrule
        Elastic             &\cite{quinlan1993elastic}          & Holonomic+    & vv, vd & circular & sim \\
        Band                &\cite{rosmann2013efficient, rosmann2015planning}   & All & vv, vd & All & sim, real \\\midrule
        \multirow{2}{*}{Spline} &\cite{mercy2016real}           & Holonomic     & vv, vd & All & sim, real \\
                            &\cite{mercy2017spline}             & All           & vv, vd & All & sim, real \\\midrule
                            &\cite{minguez2004nearness}         & Holonomic     & \-- & All & real \\
                            &\cite{ge2002dynamic}               & Holonomic, Unicycle & cv, cd & All & sim, real \\
                            &\cite{savkin2013simple}            & Unicycle      & cv, cd & All & sim, real \\
        \bottomrule
    \end{tabular}

    \caption{\label{tab:comparison_table}
    Comparison of some of the state of the art approaches for motion planning in dynamic environment\\
    holonomic+: can be applied theoretically to any vehicle type\\
    cv: constant velocity, vv: varying velocity\\ cd: constant direction, vd: varying direction\\
    sim: simulated experiments, real: experiments performed on an actual robot\\
    (Note: The experiments might have relaxed some assumptions taken for theoretical approach. 
    The \textit{Obstacle restriction} and \textit{Obstacle shape} information is based on the 
    information provided from experimental setup. If a literature provides multiple experiments 
    with different relaxed assumption, then we consider the assumption which is most resembles the theory.)
    }
\end{table}

A brief comparison between the approaches discussed in~\ref{sec:motion_planning_approaches} are shown in Table~\ref{tab:comparison_table}.
There are different criteria for comparing approaches in the field of motion planning  
in dynamic environment.
Apart from avoiding collision, the algorithms generally have at least one additional goal 
(generally reaching target in minimal time).

This target position is generally provided to the robot as problem statement, though the 
target may not be in line of sight. In this case, the robot must plan a path from its current 
position to the target location. For this task, it will need a map of the environment. 
The approaches discussed above have different assumptions regarding this parameter of 
the experiment. While~\cite{brock1999high, ogren2005convergent, seder2007dynamic,
chung2009safe,petti2005safe,hsu2002randomized,van2006anytime,ge2002dynamic} provide global 
connection to the goal, most of the approaches do not.  They can, however, be extended by 
combining with a global path planner that provides checkpoints which in line of sight of the robot.

Some approaches are only applicable to certain vehicle model types while most of them are 
applicable to every vehicle type.
There are mainly 3 types of robot models, namely holonomic (can instantaneously move in 
any direction), unicycle (can move instantaneously in the direction it is facing and can 
turn instantaneously with certain angular acceleration) and bicycle (can instantaneously 
move in the direction it is facing and can turn within certain angle depending on the linear 
velocity)\cite{hoy2015algorithms}.
Naturally, an approach which can be applied to all types of robot model is desirable. 
Though, in indoor environments, bicycle type robots might not be very useful because of 
their high turning radius. This makes them less maneuverable in narrow and cluttered spaces.

Even though an approach can be theoretically applied to any model of robot, there are some
technical reasons why that is not always desirable. For example, the approach mentioned 
in~\cite{mercy2017spline} is applicable on all models of robots and the authors show experiments
proving that it is applicable. The single iteration rate was shown to be as high as 0.98 
seconds for single moving obstacle for a differential drive robot. Whereas for a holonomic
robot with same environmental condition it was 0.4 seconds. This may increase with 
additional moving obstacles which makes this approach only suitable for holonomic robots
for a safe real time application. 

Roadmap and ICS based approaches are not dependent on the model of the robot as they 
plan in state time space of the robot. This automatically takes care of kinodynamic 
constraints during planning phase.
As stated before, for a robot motion planner must consider it own dynamic\cite{fraichard2007short}. 
Most of the methods discussed above consider kinodynamic constraints.

When the experiments are performed in a simulated environment, the information about 
the position and velocity vector of the obstacles as well as the position and heading of 
the robot are easily available. But, in real environment, these information is 
erroneous/uncertain in nature. The approaches which take this uncertainty into account are 
clearly more desirable. 
This is one of the only reason why ICS based approaches are not declared as the obvious 
choice. Even though they have the capability to reason over infinite time, they need 
perfect information about the environment to guarantee safety. On the opposite end of the 
spectrum are reactive approaches, which only reason about next iteration. What they lack 
in planning long term decision, they try to make it up on the frontier of planning with lack 
of information.

The approaches which are applicable in the environment where the obstacles are moving 
at variable velocity and in variable direction are more desirable because it resembles more 
closely to the real world. In real environment, a human might slow down to talk to someone 
or accelerate to overtake other people. All the state time space based approaches can 
theoretically deal with this constraint.

Some approaches assume only circular obstacles for the ease of calculation or for the 
experiments, but most of the approaches can be extended to consider non circular moving 
obstacles by considering more than one circle for an obstacle\cite{large2005navigation}. 
Theoretically, all state time space based approaches can deal with obstacles of any shape.

Clearly, state time space based approaches are better choice if the information about the 
environment is fairly accurate as they can deal with any robot model travelling in environment 
with obstacles of any shape travelling in variable direction at variable velocity. The 
possible problem might be the computation time required for them to plan a path in state 
time space after gathering the information.

Apart from scientific reasons, an approach is more favorable when an implementation is 
available as open source without any proprietary software. This makes its integration with
robot possible for more people and makes the approach and its claims verifiable.

