%!TEX root = ../report.tex

\chapter{Results}

% Describe results and analyse them

\section{Time comparison}%
\label{sec:time_comparison}


All planners are configured to minimise travel time. They start and end their journey at a predefined
position and have same environmental conditions (Refer~\ref{sec:setup}). The average travel time for all robots are shown
in Table~\ref{tab:avg_time_planner_comp}. 

\begin{table}[!htpb]
    \centering
    \begin{tabular}{cccccc}\toprule
        \textbf{Planner} & \textbf{Static} & \textbf{Test case} & \textbf{Test case} & \textbf{Static} & \textbf{Test case} \\
                         & \textbf{single room} & \textbf{1} & \textbf{2} & \textbf{double room}       & \textbf{3} \\\toprule
        \textbf{TEB         } & \textbf{5.205}  & 5.972  & 6.083  & \textbf{27.139} & \textbf{35.330} \\
        \textbf{Spline based} & 5.421  & \textbf{5.574}  & \textbf{5.819}  & \--    & \--    \\
        \textbf{DWA         } & 19.526 & 17.625 & 17.114 & 39.004 & \-- \\
        \textbf{EBand       } & 6.270  & 6.570  & 6.240  & 30.862 & 71.968 \\
        \textbf{EBand2      } & 6.270  & 6.025  & 6.234  & 30.862 & \-- \\
        \bottomrule
    \end{tabular}
    \caption{Average time of travel for planners for different test cases}\label{tab:avg_time_planner_comp}
\end{table}

All planners are able to traverse in test case 1 in almost
same time as their static environment counterpart (static single room). However, some planners take a 
significant amount of additional time for test case 2 and 3 than its static counterpart (static 
single room and static double room respectively). For test case 2, this can be attributed to the 
increase in number of obstacles compared to test case 1. For test case 3, there are 2 main reasons
for the large additional time. First reason is larger environment (even the smallest room in test
case 3 is 33\% larger than the room in test case 1). Second reason is that for a few seconds (4--5 
seconds every 30 seconds), complete corridor is blocked. The planner is forced to wait during this 
time until it can calculate a valid path again. 

DWA planner takes the most amount of time among
all planners in all cases. This is because the DWA approach favors circular motions which places the robot
near the goal quite fast (as fast as other planners) but it then takes a lot of time to get to the
exact goal in exact orientation. This can be seen in Figure~\ref{fig:dwa_test_case_exp_1}. The robot
is in the vicinity of the goal at \texttt{t=6.67s} but the total travel time is still 20 seconds.

EBand2 planner is slightly faster than EBand planner in test case 1 and 2. We expected the travel time to increase as mentioned
in~\ref{sub:eband2_local_planner}. The results are opposite in terms of travel time. This is
because EBand2 planner has more information about the moving obstacle than Eband planner. This 
information is velocity of the moving obstacle (implicitly calculated by \texttt{obs\_to\_costmap}).
This allows EBand2 planner to take better decisions compared to EBand planner. Eband2 chooses safer
but longer trajectory whereas EBand chooses faster trajectory but after a small duration it finds out
that the trajectory is no longer possible and it is force to wait for a short amount of time. However,
EBand planner has advantage for narrow corridors and cramped spaces because it only uses present information.

Spline based planner performs best in terms of travel time for test case 1 and 2. TEB planner performs
best in test case 3. If the spline based planner is correctly merged with a global planner, we believe
that it would have worked better than TEB in test case 3 as well. The difference in travel time 
between spline based planner and TEB planner is very small. TEB planner is considered better 
because it is able to traverse all test cases and it does that in least amount of time.


\section{Re-plan comparison}%
\label{sec:re-plan_comparison}

Number of re-plan attempts signify how well the planner processes environment information (Refer~\ref{sec:experimental_design})
We take the maximum of re-plan attempt of all experiments (Table~\ref{tab:max_re-plan_planner_comp}).
This provides the worst case scenario of 
a test case for that planner. The re-plan attempts are dependent on the initial position of the moving
obstacle. Taking the worst case scenario eliminates the dependency on that parameter for an unbiased comparison.

\begin{table}[H]
    \centering
    \begin{tabular}{cccccc}\toprule
        \textbf{Planner} & \textbf{Static} & \textbf{Test case} & \textbf{Test case} & \textbf{Static} & \textbf{Test case} \\
                         & \textbf{single room} & \textbf{1} & \textbf{2} & \textbf{double room}       & \textbf{3} \\\toprule
        \textbf{TEB         } & 0 & 0 & 1 & 0 & 4 \\
        \textbf{Spline based} & 0 & 0 & 0 & Fail & Fail \\
        \textbf{DWA         } & 0 & 0 & 2 & 0 & Fail \\
        \textbf{EBand       } & 0 & 1 & Fail & 0 & Fail \\
        \textbf{EBand2      } & 0 & 1 & 1 & 0 & Fail \\
        \bottomrule
    \end{tabular}
    \caption{Maximum re-plan attempts of planners for different test cases}\label{tab:max_re-plan_planner_comp}
\end{table}

We see a general trend of increase in number of re-plans as the environment gets more complex. TEB
planner is better than other planners in all test cases except test case 2. Spline based planner and
TEB planner have less re-plan attempts compared to other planners in most of the test cases. This is 
because these planners are able to use information about the moving obstacles directly whereas the 
other planners get this information indirectly through costmap. 

\section{Collisions comparison}%
\label{sec:collisions_comparison}

We compare average number of collisions for each planner across all experiments in Table~\ref{tab:avg_collisions_planner_comp}.

\begin{table}[H]
    \centering
    \begin{tabular}{cccccc}\toprule
        \textbf{Planner} & \textbf{Static} & \textbf{Test case} & \textbf{Test case} & \textbf{Static} & \textbf{Test case} \\
                         & \textbf{single room} & \textbf{1} & \textbf{2} & \textbf{double room}       & \textbf{3} \\\toprule
        \textbf{TEB         } & 0 & 0 & 0 & 0 & 0.75 \\
        \textbf{Spline based} & 0 & 0 & 0 & \-- & \-- \\
        \textbf{DWA         } & 0 & 0 & 0 & 0 & \-- \\
        \textbf{EBand       } & 0 & 0.25 & 0.5 & 0 & 4 \\
        \textbf{EBand2      } & 0 & 0 & 0 & 0 & \-- \\
        \bottomrule
    \end{tabular}
    \caption{Average collisions of planners for different test cases}\label{tab:avg_collisions_planner_comp}
\end{table}

TEB planner is better in this regards without any exceptions. If the global planner can be integrated
with spline based planner, we believe it might perform as good as TEB planner if not better. DWA planner performs better in test 
case 1 and test case 2, however it failed completely test case 3. After TEB planner, 
EBand planner performs better. 

\section{Overall comparison}%
\label{sec:overall_comparison}

Considering comparison of all 3 criteria, TEB planner is better than other planners. Spline based 
planner could be better if it was integrable with a global planner. EBand planner would be next better.
Depending on the application, more importance could be given to one parameter.

We have designed test case 3 to be very close to a normal corridor in
an indoor environment where people are walking in different direction and sometimes blocking the entire
corridor for a few seconds. We observer that the planners do not wait for the corridor to be unblocked
again. They either return failure or they try to pass between two obstacles and end up colliding.
Even the best planner in this comparison, TEB, tries to modify its plan again and again and shows an
oscilatory behaviour in that process.

