%
% LaTeX2e Style for MAS R&D and master thesis reports
% Author: Argentina Ortega Sainz, Hochschule Bonn-Rhein-Sieg, Germany
% Please feel free to send issues, suggestions or pull requests to:
% https://github.com/mas-group/project-report
% Based on the template created by Ronni Hartanto in 2003
%

\documentclass[thesis]{mas_report}
% \documentclass[rnd]{mas_report}

% ****************************************************
% THIS INFORMATION SHOULD BE UPDATED FOR YOUR REPORT
% ****************************************************
\author{Dharmin Bakaraniya}
\title{Robot motion planning in dynamic environment: A comparative study}
\supervisors{%
Prof.\ Dr.\ Erwin Prassler\\
Dr.\ Cesar Lopez Martinez\\
}
\date{January 2019}


% \thirdpartylogo{path/to/your/image}

\begin{document}
\begin{titlepage}
    \maketitle
\end{titlepage}

%----------------------------------------------------------------------------------------
%	PREFACE
%----------------------------------------------------------------------------------------

\pagestyle{plain}


\cleardoublepage{}
\statementpage{}

\begin{abstract}

We compare few of the state of the art approaches of motion planning in dynamic environment for
planar mobile robots.
Motion planning in static environment is a solved problem, whereas motion
planning in dynamic environment is a very complex problem. Finding out the best approach and best 
motion planner for an application would save a lot of time and money for that industry. The applications
for mobile robots would not be limited to static environment. Using a better motion planner will 
provide more safety for humans and for robots and will provide cost effective transportation of goods,
and even people!

Existing comparisons for motion planning in dynamic environments are either out of date or they do 
not compare the approaches with same parameters. The research papers proposing different approaches
do not have any benchmark comparison on their implementations. They test their approach with different 
assumptions in different environment on robots with different constraints to optimise different 
parameters. This eliminates the possibility of finding a better motion planner without implementation.
We compare few state of the art approaches
and motion planners on the basis of kinematic constraints on robot, shape and motion contraints of 
moving obstacles and experimental results of their implementation.

From an ideal motion planner, it is 
expected that plans a trajectory considering kinematic and dynamic constraints, robot geometry and
change in surrounding environment. This trajectory is expected to result in the robot reaching 
at the goal location in minimum amount of time without any collisions. Therefore, we use travel time, number of
replan attempts and number of collisions as performance measure to compare motion planner.

We create 3 test cases with increasing complexity to test 5 motion planners and compare their performance.
The motion planners are tested on a rectangular holonomic robot in a simulated environment. We find
out which motion planner is best out of the 5 planners for the designed test scenarios.

\end{abstract}


% \begin{acknowledgements}
% INCOMPLETE
% \end{acknowledgements}


\tableofcontents
\listoffigures
\listoftables

%-------------------------------------------------------------------------------
%	CONTENT CHAPTERS
%-------------------------------------------------------------------------------

\mainmatter{} % Begin numeric (1,2,3...) page numbering

\pagestyle{mainmatter}

\subfile{chapters/ch01_introduction}
\subfile{chapters/ch02_stateoftheart}
\subfile{chapters/ch03_methodology}
\subfile{chapters/ch04_solution}
\subfile{chapters/ch05_evaluation}
\subfile{chapters/ch06_results}
\subfile{chapters/ch07_conclusion}


%-------------------------------------------------------------------------------
%	APPENDIX
%-------------------------------------------------------------------------------

\begin{appendices}
\subfile{chapters/appendix}

\end{appendices}

\backmatter{}

%-------------------------------------------------------------------------------
%	BIBLIOGRAPHY
%-------------------------------------------------------------------------------
\addcontentsline{toc}{chapter}{References}
\bibliographystyle{ieeetr} % Use the plainnat bibliography style
\bibliography{../myRef.bib} % Use the bibliography.bib file as the source of references

\end{document}
